\section*{Introduction}
\addcontentsline{toc}{section}{Introduction}

L'évolution des systèmes informatiques étant en perpétuelle évolution, il est
apparu dans les années 50, le concept de ``machine learning``. Ce concept a
permis, en association avec les réseaux neuronaux apparus dans les même années,
de réaliser des systèmes en constante évolution autonome réalisant une
simulation du fonctionnement du cerveau.
Pour étudier cette association et comprendre le fonctionnement d'un neurone,
nous nous intéresserons donc dans ce projet à la définition d'un système minimal
de décision autonome.

Le but est donc la recherche d'une architecture neuronale la plus simple dont
le comportement peut être assimilé à une décision.

Il sera donc important de savoir comment sera défini la prise de décision. Pour
cela nous partirons de la définition pour en arriver à une interprétation
personnalisé à notre cas.
Nous exposerons ensuite les choix effectués lors de l'implémentation en rapport
avec cette définition de la décision.


La réalisation du projet s'est faite à l'aide du langage de programmation python.


