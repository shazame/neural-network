\section*{Introduction}
\addcontentsline{toc}{section}{Introduction}

\paragraph{}
Depuis leurs apparition, les systèmes informatiques n'ont pas cessé d'évoluer.
C'est au coeur de cet univers changeant quotidiennement qu'est apparu dans les
années 50 le concept de {\em machine learning}. En association avec les
{\em réseaux neuronaux}, celui-ci a permis de réaliser des systèmes capables
de mutation autonomes. Ces systèmes sont à la base des recherches actuelles
qui ont pour but de créer des simulations du fonctionnement du cerveau.

\paragraph{}
Pour étudier cette association et comprendre le fonctionnement d'un neurone,
nous nous intéresserons donc dans ce projet à la définition d'un système minimal
de décision autonome.
Le but est donc la recherche de l'architecture neuronale la plus simple dont
le comportement peut être assimilé à une décision.

\paragraph{}
Il sera donc important de savoir comment sera défini la prise de décision. Nous
partirons d'une définition pour aboutir à une interprétation plus spécifique au
cas que nous étudions ici.
Nous exposerons ensuite les choix effectués lors de l'implémentation et leurs
rapports avec cette définition de la décision.

\paragraph{}
La réalisation du projet s'est faite à l'aide du langage de programmation
\verb!python!.


